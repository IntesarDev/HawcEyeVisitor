\documentclass[a0,portrait]{hogent-poster}

% Info over de opleiding
\course{Graduaatsproef}
\studyprogramme{Graduaat in het Programmeren}
\academicyear{2024-2025}
\institution{Hogeschool Gent, Valentin Vaerwyckweg 1, 9000 Gent}

% Info over de bachelorproef
\title{Hawc Eye Visitor}
\subtitle{Mobiele applicatie voor bezoekers- en middelenreservaties}
\author{Intesar Al Nasralla}
\email{intesar.alnasralla@student.hogent.be}
\supervisor{Luc Vervoort}
\cosupervisor{Xavier Van Goethem}

\keywords{Hawc Eye Visitor, reserveringen, mobiel}
\projectrepo{https://dev.azure.com/graduaatsproefintesar/_git/HawcVisitor}

\begin{document}

\maketitle

\begin{abstract}
HAWC BV beschikt vandaag over een webapp voor reservaties, maar deze is vooral gericht op desktopgebruik en werkt minder vlot op smartphone. Dit project bouwt \textbf{Hawc Eye Visitor}, een mobiele applicatie waarmee gebruikers eenvoudig reservaties kunnen aanmaken, beheren, indien nodig betalen en automatisch een bevestiging via e-mail ontvangen. Het doel is een snellere en meer gebruiksvriendelijke reservatie-ervaring die beter aansluit bij dagelijks mobiel gebruik.
\end{abstract}

\begin{multicols}{2}
% -----------------------------------------------------------
\section{Doelstelling}
\begin{itemize}
  \item een gebruiksvriendelijke mobiele reservatie-app bouwen
  \item gebruikers toelaten om snel reservaties aan te maken en te beheren
  \item betalingen en bevestigingen te automatiseren
  \item een mobiele ervaring te bieden die aansluit bij dagelijks gebruik.
\end{itemize}

% -----------------------------------------------------------
\section{Functionaliteiten}
\begin{itemize}
  \item Inloggen met een persoonlijk account.
  \item Nieuwe reservatie aanmaken.
  \item Overzicht van \textit{mijn reservaties}.
  \item Reservatie annuleren binnen de afgesproken regels.
  \item Betaling uitvoeren indien vereist.
  \item Automatische bevestigingsmail via Resend.
\end{itemize}

% -----------------------------------------------------------
\section{Technologieën}
De applicatie werd ontwikkeld met:

\begin{center}
  \includegraphics[height=2.2cm]{graphics/react.png}
  \includegraphics[height=2.2cm]{graphics/firebase.png}
  \includegraphics[height=2.2cm]{graphics/ts.png}
  \includegraphics[height=2.2cm]{graphics/redux.png}
  \includegraphics[height=2.2cm]{graphics/mollie.png}
  \includegraphics[height=2.2cm]{graphics/git.png}
  \includegraphics[height=2.2cm]{graphics/vercel.png}
\end{center}

% -----------------------------------------------------------
\section{Interface}
Onderstaande figuren tonen enkele schermen van de mobiele applicatie:
\begin{center}
  \captionsetup{type=figure}
  \includegraphics[width=0.70\linewidth]{graphics/screens.png}
  \captionof{figure}{Screens van login- en reservatiescherm binnen Hawc Eye Visitor.}
\end{center}

% -----------------------------------------------------------
\section{Toekomstig werk}
\begin{itemize}
  \item Integratie met het echte backend-systeem van HAWC BV.
  \item Extra beheersfuncties voor interne medewerkers.
\end{itemize}

% -----------------------------------------------------------
\section{Conclusie}
Hawc Eye Visitor biedt een gebruiksvriendelijke en moderne oplossing voor het beheren van reservaties via smartphone. Het project toont aan hoe een mobiele aanpak de toegankelijkheid en snelheid van reserveren binnen HAWC BV aanzienlijk kan verbeteren.

\end{multicols}

\end{document}
